\section{Introduction}

\begin{frame}
 \frametitle{Introduction}

 \begin{itemize}
  \item CIG is transitioning code from Subversion to Git
  \item Also shifting most hosting to \href{https://github.com/}{GitHub}
  \item All Seismology code has been migrated:
   \begin{itemize}
    \item \href{https://github.com/geodynamics/specfem1d}{Specfem1D}
    \item \href{https://github.com/geodynamics/specfem2d}{Specfem2D}
    \item \href{https://github.com/geodynamics/specfem3d}{Specfem3D}
    \item \href{https://github.com/geodynamics/specfem3d_globe}{Specfem3D Globe}
    \item \href{https://github.com/geodynamics/mineos}{Mineos}
    \item \href{https://github.com/geodynamics/flexwin}{Flexwin}
    \item \href{https://github.com/geodynamics/axisem}{AxiSEM}
   \end{itemize}
  \item Also code for
        \href{http://geodynamics.org/cig/software/\#cs}{Computational Science},
        \href{http://geodynamics.org/cig/software/\#geodyn}{Geodynamo},
        \href{http://geodynamics.org/cig/software/\#long}{Long-Term Tectonics},
        \href{http://geodynamics.org/cig/software/\#mc}{Mantle Convection},
        \href{http://geodynamics.org/cig/software/\#short}{Short-Term Crustal
        Dynamics}
 \end{itemize}
\end{frame}

\subsection{What is Git?}

\begin{frame}
 \frametitle{Introduction}
 \framesubtitle{What is Git?}

 The stupid content tracker\footnotemark
 \begin{itemize}
  \item \href{http://git-scm.com/about/distributed}{Distributed}
   \begin{itemize}
    \item Multiple backups
    \item Flexible workflow
   \end{itemize}
  \item \href{http://git-scm.com/about/branching-and-merging}{Quick and easy
        branching}
   \begin{itemize}
    \item Frictionless context switching
    \item Role-based codelines
    \item Feature based workflow
    \item Disposable experimentation
   \end{itemize}
  \item \href{http://git-scm.com/about/small-and-fast}{Small and fast}
        --- One or two orders of magnitude faster than SVN (except cloning)
  \item \href{http://git-scm.com/about/info-assurance}{Repository integrity}
        --- Files, commit messages, dates
  \item \href{http://git-scm.com/about/staging-area}{Staging area}
        --- Flexible committing
  \item \href{http://git-scm.com/about/free-and-open-source}{Free and open
        source} --- GNU GPL 2.0
 \end{itemize}

 \footnotetext{\url{http://git-scm.com/about/}}
\end{frame}

\subsection{What is GitHub?}

\begin{frame}
 \frametitle{Introduction}
 \framesubtitle{What is GitHub?}

 Provides hosting and resources for open source software:
 \begin{itemize}
  \item Issue tracking
   \begin{itemize}
    \item Tasks
    \item Features
    \item Pull requests
    \item etc.
   \end{itemize}
  \item Code review
   \begin{itemize}
    \item Commit comments,
    \item Pull requests,
    \item Diffs,
    \item etc.
   \end{itemize}
  \item Wikis
  \item Collaborative tools
  \item Statistics
 \end{itemize}
\end{frame}

\subsection{Version Control Basics}

\begin{frame}
 \frametitle{Version Control Concepts}

 \begin{description}[Operations]
  \item[Repository] Complete copy of entire history of project\\
                    Files, Changes, Author names, Dates, etc.
  \item[Commit] Atomic record of changes to files, Author (and committer) of
                change, Date
  \item[Branch] Name for a commit that follows new commits (e.g.,
                amazing-new-feature)
  \item[Tag] Immovable name for a commit (e.g., v1.0)
  \item[Operations]
   \begin{itemize}
    \item Clone --- Download a copy of history
    \item Commit --- Record your changes
    \item Fetch --- Download new changes
    \item Merge --- Combine new changes with existing files
    \item \ldots
   \end{itemize}
 \end{description}
\end{frame}

\begin{frame}
 \frametitle{Version Control Concepts with Subversion}

 \vfill
 \begin{center}
  {\LARGE \alert{Ignore} Subversion if possible} \\
  {\small (Next three slides for those who know it already)}
 \end{center}
 \vfill
\end{frame}

%
% SVN Checkout
%

\begin{frame}
 \frametitle{Version Control Concepts with Subversion}
 \framesubtitle{Checkout}

 \begin{columns}
  \begin{column}{0.3\textwidth}
   \begin{tikzpicture}[
      remember picture,
      show background rectangle,
      commit/.style={draw, rectangle, rounded corners, fill=solarizedRebase02,
                     inner sep=1pt, minimum width=1.5em},
      connections/.style={draw}
    ]
    \matrix [column sep={1em,between origins}, row sep={1.5em,between origins},
             ampersand replacement=\&]%
    {
     \ghost{name}; \\
     \ghost{10}; \\
     \node[commit, visible on=<3->] (9) {9}; \\
     \node[commit, visible on=<3->] (8) {8}; \\
     \node[commit] (7) {7}; \\
     \node[commit] (6) {6}; \\
     \node[commit] (5) {5}; \\
     \node[commit] (4) {4}; \\
     \node[commit] (3) {3}; \\
     \node[commit] (2) {2}; \\
     \ghost{1}; \\
    };
    \node[anchor=south, align=center] at (name) {Central\\Repository};
    \foreach \x in {3,...,7} {
     \pgfmathparse{int(\x-1)}
     \connect{\pgfmathresult}{\x};
    }
    \path[connections, densely dashed] (1) -- (2);
    \path[connections, visible on=<3->] (7) -- (8);
    \path[connections, visible on=<3->] (8) -- (9);
   \end{tikzpicture}
  \end{column}
  \begin{column}{0.6\textwidth}
   \begin{tikzpicture}[%
      remember picture,
      show background rectangle,
      background rectangle/.style={draw, visible on=<4->},
      visible on=<4->,
      every node/.style={anchor=west, minimum height=1em,
                         font={\scriptsize\ttfamily}, inner sep=2.5pt},
      every edge/.style={thick},
      root/.style={},
      grow via three points={one child at (0.25,-1em) and
                             two children at (0.25,-1em) and (0.25,-1.75em)},
      edge from parent path={(\tikzparentnode.south) |- (\tikzchildnode.west)}
    ]
    \node [root] (E root) {specfem3d\_globe}
     child { node {DATA} }
     child { node {doc} }
     child { node {EXAMPLES} }
     child { node {setup} }
     child { node {src} }
     child { node {utils} }
     child { node {\dots} };
    \node[anchor=south, font={\sffamily}, align=center] at (E root.north)
     {Elliott's\\Working Directory};
   \end{tikzpicture}

   \hfill
   \begin{tikzpicture}[%
      remember picture,
      show background rectangle,
      background rectangle/.style={draw, visible on=<2->},
      visible on=<2->,
      every node/.style={anchor=west, minimum height=1em,
                         font={\scriptsize\ttfamily}, inner sep=2.5pt},
      every edge/.style={thick},
      root/.style={},
      grow via three points={one child at (0.25,-1em) and
                             two children at (0.25,-1em) and (0.25,-1.75em)},
      edge from parent path={(\tikzparentnode.south) |- (\tikzchildnode.west)}
    ]
    \node [root] (P root) {specfem3d\_globe}
     child { node {DATA} }
     child { node {doc} }
     child { node {EXAMPLES} }
     child { node {setup} }
     child { node {src} }
     child { node {utils} }
     child { node {\dots} };
    \node[anchor=south, font={\sffamily}, align=center] at (P root.north)
     {Piero's\\Working Directory};
   \end{tikzpicture}
  \end{column}
 \end{columns}

 \begin{tikzpicture}[
    remember picture, overlay,
    path/.style={thick, ->},
    label/.style={decorate, decoration={text along path, text align=center,
                                        text color=solarizedRebase0,
                                        raise=1pt,
                                        text={svn checkout}}}
  ]
  \draw<2>[path] (7.east) to[out=0,in=180] (P root.west);
  \draw<2>[label] (7.east) to[out=0,in=180] (P root.west);
  \draw<3->[path] (P root.west) to[out=180,in=0] (7.east);
  \draw<4->[path] (9.east) to[out=0,in=180] (E root.west);
  \draw<4->[label] (9.east) to[out=0,in=180] (E root.west);
 \end{tikzpicture}
\end{frame}

%
% SVN Commit
%

\begin{frame}
 \frametitle{Version Control Concepts with Subversion}
 \framesubtitle{Commits}

 \begin{columns}
  \begin{column}{0.3\textwidth}
   \begin{tikzpicture}[
      remember picture,
      show background rectangle,
      commit/.style={draw, rectangle, rounded corners, fill=solarizedRebase02,
                     inner sep=1pt, minimum width=1.5em},
      connections/.style={draw}
    ]
    \matrix [column sep={1em,between origins}, row sep={1.5em,between origins},
             ampersand replacement=\&]%
    {
     \ghost{name}; \\
     \node[commit, visible on=<3->] (10) {10}; \\
     \node[commit] (9) {9}; \\
     \node[commit] (8) {8}; \\
     \node[commit] (7) {7}; \\
     \node[commit] (6) {6}; \\
     \node[commit] (5) {5}; \\
     \node[commit] (4) {4}; \\
     \node[commit] (3) {3}; \\
     \node[commit] (2) {2}; \\
     \ghost{1}; \\
    };
    \node[anchor=south, align=center] at (name) {Central\\Repository};
    \foreach \x in {3,...,9} {
     \pgfmathparse{int(\x-1)}
     \connect{\pgfmathresult}{\x};
    }
    \path[connections, densely dashed] (1) -- (2);
    \path[connections, visible on=<3->] (9) -- (10);
   \end{tikzpicture}
  \end{column}
  \begin{column}{0.6\textwidth}
   \begin{tikzpicture}[%
      remember picture,
      show background rectangle,
      every node/.style={anchor=west, minimum height=1em,
                         font={\scriptsize\ttfamily}, inner sep=2.5pt},
      every edge/.style={thick},
      root/.style={},
      grow via three points={one child at (0.25,-1em) and
                             two children at (0.25,-1em) and (0.25,-1.75em)},
      edge from parent path={(\tikzparentnode.south) |- (\tikzchildnode.west)}
    ]
    \node [root] (E root) {specfem3d\_globe}
     child { node {DATA} }
     child { node {doc} }
     child { node {EXAMPLES} }
     child { node {setup} }
     child { node {src} }
     child { node {utils} }
     child { node {\dots} };
    \node[anchor=south, font={\sffamily}, align=center] at (E root.north)
     {Elliott's\\Working Directory};
   \end{tikzpicture}

   \hfill
   \begin{tikzpicture}[%
      remember picture,
      show background rectangle,
      background rectangle/.style={draw,visible on=<2-3>},
      visible on=<2-3>,
      every node/.style={anchor=west, minimum height=1em,
                         font={\scriptsize\ttfamily}, inner sep=2.5pt},
      every edge/.style={thick},
      root/.style={},
      grow via three points={one child at (0.25,-1em) and
                             two children at (0.25,-1em) and (0.25,-1.75em)},
      edge from parent path={(\tikzparentnode.south) |- (\tikzchildnode.west)}
    ]
    \node [root] (M root) {specfem3d\_globe}
     child { node {DATA} }
     child { node {doc} }
     child { node {EXAMPLES} }
     child { node {setup} }
     child { node {\itshape src} }
     child { node {utils} }
     child { node {\dots} };
    \node[anchor=south, font={\sffamily}, align=center] at (M root.north)
     {Elliott's Modified\\Working Directory};
   \end{tikzpicture}
  \end{column}
 \end{columns}

 \begin{tikzpicture}[
    remember picture, overlay,
    path/.style={thick, ->},
    label/.style={decorate, decoration={text along path, text align=center,
                                        text color=solarizedRebase0}}
  ]
  \draw<1-2>[path] (E root.west) to[out=180,in=0] (9.east);
  \draw<2-3>[path] (E root.east) to[out=0,in=0] (M root.east);
  \draw<2>[label, decoration={raise=1pt, text={Editing}}]
   (E root.east) to[out=0,in=0] (M root.east);
  \draw<3>[path, dashed] (E root.west) to[out=180,in=0] (9.east);
  \draw<3>[path] (M root.west) to[out=180,in=0] (10.east);
  \draw<3>[label, decoration={raise=-1em, text={svn commit}}]
   (10.east) to[out=0,in=180] (M root.west);
  \draw<4->[path] (E root.west) to[out=180,in=0] (10.east);
 \end{tikzpicture}
\end{frame}

%
% SVN Update
%

\begin{frame}
 \frametitle{Version Control Concepts with Subversion}
 \framesubtitle{Updating}

 \begin{columns}
  \begin{column}{0.3\textwidth}
   \begin{tikzpicture}[
      remember picture,
      show background rectangle,
      commit/.style={draw, rectangle, rounded corners, fill=solarizedRebase02,
                     inner sep=1pt, minimum width=1.5em},
      connections/.style={draw}
    ]
    \matrix [column sep={1em,between origins}, row sep={1.5em,between origins},
             ampersand replacement=\&]%
    {
     \ghost{name}; \\
     \node[commit] (10) {10}; \\
     \node[commit] (9) {9}; \\
     \node[commit] (8) {8}; \\
     \node[commit] (7) {7}; \\
     \node[commit] (6) {6}; \\
     \node[commit] (5) {5}; \\
     \node[commit] (4) {4}; \\
     \node[commit] (3) {3}; \\
     \node[commit] (2) {2}; \\
     \ghost{1}; \\
    };
    \node[anchor=south, align=center] at (name) {Central\\Repository};
    \foreach \x in {3,...,10} {
     \pgfmathparse{int(\x-1)}
     \connect{\pgfmathresult}{\x};
    }
    \path[connections, densely dashed] (1) -- (2);
   \end{tikzpicture}
  \end{column}
  \begin{column}{0.6\textwidth}
   \vfill

   \hfill
   \begin{tikzpicture}[%
      remember picture,
      show background rectangle,
      every node/.style={anchor=west, minimum height=1em,
                         font={\scriptsize\ttfamily}, inner sep=2.5pt},
      every edge/.style={thick},
      root/.style={},
      grow via three points={one child at (0.25,-1em) and
                             two children at (0.25,-1em) and (0.25,-1.75em)},
      edge from parent path={(\tikzparentnode.south) |- (\tikzchildnode.west)}
    ]
    \node [root] (P root) {specfem3d\_globe}
     child { node {DATA} }
     child { node {doc} }
     child { node {EXAMPLES} }
     child { node {setup} }
     child { node {src} }
     child { node {utils} }
     child { node {\dots} };
    \node[anchor=south, font={\sffamily}, align=center] at (P root.north)
     {Piero's\\Working Directory};
   \end{tikzpicture}
  \end{column}
 \end{columns}

 \begin{tikzpicture}[
    remember picture, overlay,
    path/.style={thick, ->},
    label/.style={decorate, decoration={text along path, text align=center,
                                        text color=solarizedRebase0,
                                        raise=1pt,
                                        text={svn update}}}
  ]
  \draw<1>[path] (P root.west) to[out=180,in=0] (7.east);
  \draw<2>[path, dashed] (P root.west) to[out=180,in=0] (7.east);
  \draw<2>[path] (10.east) to[out=0,in=180] (P root.west);
  \draw<2>[label] (10.east) to[out=0,in=180] (P root.west);
  \draw<3>[path] (P root.west) to[out=180,in=0] (10.east);
 \end{tikzpicture}
\end{frame}

\begin{frame}
 \frametitle{Version Control Concepts with Git}

 \vfill
 \begin{center}
  {\LARGE \alert{Forget} Subversion if possible} \\
  {\small (Previous slides for those who know it already)}
 \end{center}
 \vfill
\end{frame}

%
% Git Cloning
%

\begin{frame}
 \frametitle{Version Control Concepts with Git}
 \framesubtitle{Cloning}

 \begin{columns}
  \begin{column}{0.3\textwidth}
   \begin{tikzpicture}[
     remember picture,
     show background rectangle,
     commit/.style={draw, rectangle, rounded corners, fill=solarizedRebase02,
                    inner sep=1pt, minimum width=1.5em},
     connections/.style={draw}
    ]
    \matrix [column sep={1em,between origins}, row sep={1.5em,between origins},
             ampersand replacement=\&]%
    {
     \ghost{name}; \\
     \ghost{10}; \\
     \node[commit, visible on=<3->] (9) {9}; \\
     \node[commit, visible on=<3->] (8) {8}; \\
     \node[commit] (7) {7}; \\
     \node[commit] (6) {6}; \\
     \node[commit] (5) {5}; \\
     \node[commit] (4) {4}; \\
     \node[commit] (3) {3}; \\
     \node[commit] (2) {2}; \\
     \ghost{1}; \\
    };
    \node[anchor=south, align=center] (repository) at (name)
     {Central\\Repository};
    \foreach \x in {3,...,7} {
     \pgfmathparse{int(\x-1)}
     \connect{\pgfmathresult}{\x};
    }
    \path[connections, densely dashed] (1) -- (2);
    \path[connections, visible on=<3->] (7) -- (8);
    \path[connections, visible on=<3->] (8) -- (9);
   \end{tikzpicture}
  \end{column}
  \begin{column}{0.6\textwidth}
   \begin{tikzpicture}[%
     remember picture,
     show background rectangle,
     background rectangle/.style={draw, visible on=<4->},
     visible on=<4->
    ]
    \begin{scope}[
      commit/.style={draw, circle, fill=solarizedRebase02,
                     inner sep=0.5pt, minimum width=0.5em,
                     font={\tiny\sffamily}},
      connections/.style={draw}
     ]
     \matrix [column sep={1em,between origins}, row sep={1em,between origins},
              ampersand replacement=\&]%
     {
      \node[commit] (E 9) {9}; \\
      \node[commit] (E 8) {8}; \\
      \node[commit] (E 7) {7}; \\
      \node[commit] (E 6) {6}; \\
      \node[commit] (E 5) {5}; \\
     };
     \node[anchor=south, align=center, font={\scriptsize}] at (E 9.north)
      (E repository) {Local\\Repository};
     \foreach \x in {6,...,9} {
      \pgfmathparse{int(\x-1)}
      \connect{E \pgfmathresult}{E \x};
     }
     \path[connections, densely dotted] (E 5.south) -- +(0,-0.5em);
    \end{scope}
    \begin{scope}[
      every node/.style={anchor=west, minimum height=1em,
                         font={\tiny\ttfamily}, inner sep=2.5pt},
      every edge/.style={thick},
      root/.style={},
      grow via three points={one child at (0.25,-0.75em) and
                             two children at (0.25,-.75em) and (0.25,-1.25em)},
      edge from parent path={(\tikzparentnode.south) |- (\tikzchildnode.west)}
     ]
     \node[anchor=west, font={\scriptsize\sffamily}, align=center, xshift=1em]
      at (E repository.east) (E work) {Working\\Directory};
     \node [root, anchor=north] at (E work.south) (E root) {specfem3d\_globe}
      child { node {DATA} }
      child { node {doc} }
      child { node {EXAMPLES} }
      child { node {setup} }
      child { node {src} }
      child { node {utils} }
      child { node {\dots} };
    \end{scope}
    \node[anchor=south, shift={(0.5em,1em)}] at (E repository.east)
     {Elliott's Computer};
    \draw[->] (E 9.east) to[out=0,in=180] (E root.west);
   \end{tikzpicture}

   \hfill
   \begin{tikzpicture}[%
     remember picture,
     show background rectangle,
     background rectangle/.style={draw, visible on=<2->},
     visible on=<2->
    ]
    \begin{scope}[
      commit/.style={draw, circle, fill=solarizedRebase02,
                     inner sep=0.5pt, minimum width=0.5em,
                     font={\tiny\sffamily}},
      connections/.style={draw}
     ]
     \matrix [column sep={1em,between origins}, row sep={1em,between origins},
              ampersand replacement=\&]%
     {
      \ghost{P 9}; \\
      \ghost{P 8}; \\
      \node[commit] (P 7) {7}; \\
      \node[commit] (P 6) {6}; \\
      \node[commit] (P 5) {5}; \\
     };
     \node[anchor=south, align=center, font={\scriptsize}] at (P 9.north)
      (P repository) {Local\\Repository};
     \foreach \x in {6,...,7} {
      \pgfmathparse{int(\x-1)}
      \connect{P \pgfmathresult}{P \x};
     }
     \path[connections, densely dotted] (P 5.south) -- +(0,-0.5em);
    \end{scope}
    \begin{scope}[
      every node/.style={anchor=west, minimum height=1em,
                         font={\tiny\ttfamily}, inner sep=2.5pt},
      every edge/.style={thick},
      root/.style={},
      grow via three points={one child at (0.25,-0.75em) and
                             two children at (0.25,-0.75em) and (0.25,-1.25em)},
      edge from parent path={(\tikzparentnode.south) |- (\tikzchildnode.west)}
     ]
     \node[anchor=west, font={\scriptsize\sffamily}, align=center, xshift=1em]
      at (P repository.east) (P work) {Working\\Directory};
     \node [root, anchor=north] at (P work.south) (P root) {specfem3d\_globe}
      child { node {DATA} }
      child { node {doc} }
      child { node {EXAMPLES} }
      child { node {setup} }
      child { node {src} }
      child { node {utils} }
      child { node {\dots} };
    \end{scope}
    \node[anchor=south, shift={(0.5em,1em)}] at (P repository.east)
     {Piero's Computer};
    \draw<2->[->] (P 7.east) to[out=0,in=180] (P root.west);
   \end{tikzpicture}
  \end{column}
 \end{columns}

 \begin{tikzpicture}[
   remember picture, overlay,
   path/.style={thick, ->},
   label/.style={decorate, decoration={text along path, text align=center,
                                       text color=solarizedRebase0,
                                       raise=1pt,
                                       text={git clone}}}
  ]
  \draw<2>[path] (repository) to[out=0,in=180] (P repository.west);
  \draw<2>[label] (repository) to[out=0,in=180] (P repository.west);
  \draw<3->[path,|-|] (repository) to[out=0,in=180] (P repository.west);
  \draw<4->[path] (repository) to[out=0,in=180] (E repository.west);
  \draw<4->[label] (repository) to[out=0,in=180] (E repository.west);
 \end{tikzpicture}
\end{frame}

%
% Git Commit
%

\begin{frame}
 \frametitle{Version Control Concepts with Git}
 \framesubtitle{Commits}

 \begin{columns}
  \begin{column}{0.3\textwidth}
   \begin{tikzpicture}[
     remember picture,
     show background rectangle,
     commit/.style={draw, rectangle, rounded corners, fill=solarizedRebase02,
                    inner sep=1pt, minimum width=1.5em},
     connections/.style={draw}
    ]
    \matrix [column sep={1em,between origins}, row sep={1.5em,between origins},
             ampersand replacement=\&]%
    {
     \ghost{name}; \\
     \node[commit, visible on=<5->] (10) {10}; \\
     \node[commit] (9) {9}; \\
     \node[commit] (8) {8}; \\
     \node[commit] (7) {7}; \\
     \node[commit] (6) {6}; \\
     \node[commit] (5) {5}; \\
     \node[commit] (4) {4}; \\
     \node[commit] (3) {3}; \\
     \node[commit] (2) {2}; \\
     \ghost{1}; \\
    };
    \node[anchor=south,align=center] (repository) at (name)
     {Central\\Repository};
    \foreach \x in {3,...,9} {
     \pgfmathparse{int(\x-1)}
     \connect{\pgfmathresult}{\x};
    }
    \path[connections, densely dashed] (1) -- (2);
    \path[connections, visible on=<5->] (9) -- (10);
   \end{tikzpicture}
  \end{column}
  \begin{column}{0.6\textwidth}
   \begin{tikzpicture}[%
      remember picture,
      show background rectangle
    ]
    \begin{scope}[
      commit/.style={draw, rectangle, rounded corners, fill=solarizedRebase02,
                     inner sep=1pt, minimum height=0.75em, minimum width=1em,
                     font={\tiny\sffamily}},
      connections/.style={draw}
     ]
     \matrix [column sep={1em,between origins}, row sep={1.2em,between origins},
              ampersand replacement=\&]%
     {
      \node[commit, visible on=<3->] (E 10) {10}; \\
      \node[commit] (E 9) {9}; \\
      \node[commit] (E 8) {8}; \\
      \node[commit] (E 7) {7}; \\
      \node[commit] (E 6) {6}; \\
      \node[commit] (E 5) {5}; \\
      \ghost{E 4}; \\
     };
     \node[anchor=south, align=center, font={\scriptsize}] at (E 10.north)
      (E repository) {Local\\Repository};
     \foreach \x in {6,...,9} {
      \pgfmathparse{int(\x-1)}
      \connect{E \pgfmathresult}{E \x};
     }
     \path[connections, densely dotted] (E 4) -- (E 5);
     \path[connections, visible on=<3->] (E 9) -- (E 10);
    \end{scope}
    \begin{scope}[
      every node/.style={anchor=west, minimum height=1em,
                         font={\tiny\ttfamily}, inner sep=2.5pt},
      every edge/.style={thick},
      root/.style={},
      grow via three points={one child at (0.25,-0.75em) and
                             two children at (0.25,-0.75em) and (0.25,-1.25em)},
      edge from parent path={(\tikzparentnode.south) |- (\tikzchildnode.west)}
     ]
     \node[anchor=west, font={\scriptsize\sffamily}, align=center, xshift=1em]
      at (E repository.east) (E work) {Working\\Directory};
     \node [root, anchor=north] at (E work.south) (E root) {specfem3d\_globe}
      child { node {DATA} }
      child { node {doc} }
      child { node {EXAMPLES} }
      child { node {setup} }
      child { node {src} }
      child { node {utils} }
      child { node (E last) {\dots} };
    \end{scope}
    \begin{scope}[
      visible on=<2-3>,
      every node/.style={anchor=west, minimum height=1em,
                         font={\tiny\ttfamily}, inner sep=2.5pt},
      every edge/.style={thick},
      root/.style={},
      label/.style={decorate, decoration={text along path, text align=center,
                                          text color=solarizedRebase0}},
      grow via three points={one child at (0.25,-0.75em) and
                             two children at (0.25,-0.75em) and (0.25,-1.25em)},
      edge from parent path={(\tikzparentnode.south) |- (\tikzchildnode.west)}
     ]
    \node[anchor=north, font={\scriptsize\sffamily}, align=center, xshift=1em]
     at (E last.south) (M work) {Modified\\Working Directory};
    \node [root, anchor=north] at (M work.south) (M root) {specfem3d\_globe}
     child { node {DATA} }
     child { node {doc} }
     child { node {EXAMPLES} }
     child { node {setup} }
     child { node {\itshape src} }
     child { node {utils} }
     child { node {\dots} };
    \end{scope}

    \draw<1-2>[->] (E root.west) to[out=180,in=0] (E 9.east);
    \draw<3>[->, dashed] (E root.west) to[out=180,in=0] (E 9.east);
    \draw<4->[->] (E root.west) to[out=180,in=0] (E 10.east);

    \node[anchor=south, shift={(1em,1em)}] at (E repository.east)
     {Elliott's Computer};
   \end{tikzpicture}
  \end{column}
 \end{columns}

 \begin{tikzpicture}[
    remember picture, overlay,
    path/.style={thick, ->},
    label/.style={decorate, decoration={text along path, text align=center,
                                        text color=solarizedRebase0}}
  ]
  \draw<1-4>[path,|-|] (repository.east) to[out=0,in=180] (E repository.west);
  \draw<2-3>[->] (E root.east) to[out=0,in=0] (M root.east);
  \draw<2>[label, decoration={raise=1pt,
                              text={|\scriptsize| Editing~~~~~}}]
   (E root.east) to[out=0,in=0] (M root.east);
  \draw<3>[->] (M root.west) to[out=180,in=0] (E 10.east);
  \draw<3>[label, decoration={raise=-0.75em,
                              text={|\scriptsize| git commit}}]
   (E 10.east) to[out=0,in=180] (M root.west);
  \draw<5>[path] (E repository.west) to[out=180,in=0] (repository.east);
  \draw<5>[label, decoration={raise=1pt, text={git push}}]
   (repository.east) to[out=0,in=180] (E repository.west);
 \end{tikzpicture}
\end{frame}

%
% Git Updating
%

\begin{frame}
 \frametitle{Version Control Concepts with Git}
 \framesubtitle{Updating}

 \begin{columns}
  \begin{column}{0.3\textwidth}
   \begin{tikzpicture}[
     remember picture,
     show background rectangle,
     commit/.style={draw, rectangle, rounded corners, fill=solarizedRebase02,
                    inner sep=1pt, minimum width=1.5em},
     connections/.style={draw}
    ]
    \matrix [column sep={1em,between origins}, row sep={1.5em,between origins},
             ampersand replacement=\&]%
    {
     \ghost{name}; \\
     \node[commit] (10) {10}; \\
     \node[commit] (9) {9}; \\
     \node[commit] (8) {8}; \\
     \node[commit] (7) {7}; \\
     \node[commit] (6) {6}; \\
     \node[commit] (5) {5}; \\
     \node[commit] (4) {4}; \\
     \node[commit] (3) {3}; \\
     \node[commit] (2) {2}; \\
     \ghost{1}; \\
    };
    \node[anchor=south, align=center] (repository) at (name)
     {Central\\Repository};
    \foreach \x in {3,...,10} {
     \pgfmathparse{int(\x-1)}
     \connect{\pgfmathresult}{\x};
    }
    \path[connections, densely dashed] (1) -- (2);
   \end{tikzpicture}
  \end{column}
  \begin{column}{0.6\textwidth}
   \vfill

   \hfill
   \begin{tikzpicture}[%
     remember picture,
     show background rectangle,
    ]
    \begin{scope}[
      commit/.style={draw, rectangle, rounded corners, fill=solarizedRebase02,
                     inner sep=1pt, minimum height=0.75em, minimum width=1em,
                     font={\tiny\sffamily}},
      connections/.style={draw}
     ]
     \matrix [column sep={1em,between origins}, row sep={1em,between origins},
              ampersand replacement=\&]%
     {
      \node[commit, visible on=<2->] (P 10) {10}; \\
      \node[commit, visible on=<2->] (P 9) {9}; \\
      \node[commit, visible on=<2->] (P 8) {8}; \\
      \node[commit] (P 7) {7}; \\
      \node[commit] (P 6) {6}; \\
      \node[commit] (P 5) {5}; \\
      \ghost{P 4}; \\
     };
     \node[anchor=south, align=center, font={\scriptsize}] at (P 10.north)
      (P repository) {Local\\Repository};
     \foreach \x in {6,...,7} {
      \pgfmathparse{int(\x-1)}
      \connect{P \pgfmathresult}{P \x};
     }
     \path[connections, densely dotted] (P 4) -- (P 5);
     \path[connections, visible on=<2->] (P 7) -- (P 8);
     \path[connections, visible on=<2->] (P 8) -- (P 9);
     \path[connections, visible on=<2->] (P 9) -- (P 10);
    \end{scope}
    \begin{scope}[
      every node/.style={anchor=west, minimum height=1em,
                         font={\tiny\ttfamily}, inner sep=2.5pt},
      every edge/.style={thick},
      root/.style={},
      grow via three points={one child at (0.25,-0.75em) and
                             two children at (0.25,-0.75em) and (0.25,-1.25em)},
      edge from parent path={(\tikzparentnode.south) |- (\tikzchildnode.west)}
     ]
     \node[anchor=west, font={\scriptsize\sffamily}, align=center, xshift=3em]
      at (P repository.east) (P work) {Working\\Directory};
     \node [root, anchor=north] at (P work.south) (P root) {specfem3d\_globe}
      child { node {DATA} }
      child { node {doc} }
      child { node {EXAMPLES} }
      child { node {setup} }
      child { node {src} }
      child { node {utils} }
      child { node {\dots} };
    \end{scope}
    \node[anchor=south, shift={(1.5em,1em)}] at (P repository.east)
     {Piero's Computer};

    \draw<1-2>[->] (P root.west) to[out=180,in=0] (P 7.east);
    \draw<3>[->, dashed] (P root.west) to[out=180,in=0] (P 7.east);
    \draw<3>[->] (P 10.east) to[out=0,in=180] (P root.west);
    \draw<4>[->] (P root.west) to[out=180,in=0] (P 10.east);
   \end{tikzpicture}
  \end{column}
 \end{columns}

 \begin{tikzpicture}[
   remember picture, overlay,
   path/.style={thick, ->},
   label/.style={decorate, decoration={text along path, text align=center,
                                       text color=solarizedRebase0, raise=1pt}}
  ]
  \draw<1>[path,|-|] (repository) to[out=0,in=180] (P repository.west);
  \draw<2>[path] (repository) to[out=0,in=180] (P repository.west);
  \draw<2>[label, decoration={text={git fetch}}]
   (repository) to[out=0,in=180] (P repository.west);
  \draw<3->[path,|-|] (repository) to[out=0,in=180] (P repository.west);
  \draw<3>[label, decoration={text={|\scriptsize| git merge}}]
   (P 10.east) to[out=0,in=180] (P root.west);
 \end{tikzpicture}
\end{frame}

\begin{frame}
 \frametitle{Git Basics}
 \framesubtitle{Commits}

 \begin{itemize}
  \item Denoted by SHA1 hash, e.g., cd83b686a7c6014e132933f685157d43151a665e
        (can be shortened); based on:
   \begin{itemize}
    \item Tree of files
    \item Parent commit(s)
    \item Commit message
    \item Author name/email/timestamp
    \item Committer name/email/timestamp
   \end{itemize}
  \pause
  \item Form a Directed Acyclic Graph, referenced by:
   \begin{itemize}
    \item Child commits
    \item Branches
    \item Tags
    \item HEAD (Working Directory link)
    \item The reflog (advanced, but useful at times)
   \end{itemize}
 \end{itemize}
\end{frame}

\begin{frame}
 \frametitle{Git Basics}
 \framesubtitle{Remotes}

 Remotes are references to other repositories; URIs can be:
 \begin{itemize}
  \item HTTPS:\\
    {\small \url{https://github.com/geodynamics/specfem3d_globe.git}}
  \item SSH:\\
    \url{git@github.com:geodynamics/specfem3d_globe.git}
  \item File:\\
    \url{file:///path/to/specfem3d_globe}
  \item Local Path:\\
    \url{/home/elliott/specfem3d_globe}
 \end{itemize}
\end{frame}

\begin{frame}
 \frametitle{Git Basics}
 \framesubtitle{Remotes}

 \begin{columns}
  \begin{column}{0.3\textwidth}
   \begin{tikzpicture}[
     remember picture,
     show background rectangle,
     commit/.style={draw, rectangle, rounded corners, fill=solarizedRebase02,
                    inner sep=1pt, minimum width=1.5em},
     connections/.style={draw}
    ]
    \matrix [column sep={1em,between origins}, row sep={1.5em,between origins},
             ampersand replacement=\&]%
    {
     \ghost{name}; \\
     \ghost{10}; \\
     \node[commit] (9) {9}; \\
     \node[commit] (8) {8}; \\
     \node[commit] (7) {7}; \\
     \node[commit] (6) {6}; \\
     \node[commit] (5) {5}; \\
     \node[commit] (4) {4}; \\
     \node[commit] (3) {3}; \\
     \node[commit] (2) {2}; \\
     \ghost{1}; \\
    };
    \node[anchor=south, align=center] (repository) at (name)
     {Central\\Repository};
    \foreach \x in {3,...,9} {
     \pgfmathparse{int(\x-1)}
     \connect{\pgfmathresult}{\x};
    }
    \path[connections, densely dashed] (1) -- (2);
   \end{tikzpicture}
  \end{column}
  \begin{column}{0.6\textwidth}
   \begin{tikzpicture}[%
     remember picture,
     show background rectangle
    ]
    \begin{scope}[
      commit/.style={draw, circle, fill=solarizedRebase02,
                     inner sep=0.5pt, minimum width=0.5em,
                     font={\tiny\sffamily}},
      connections/.style={draw}
     ]
     \matrix [column sep={1em,between origins}, row sep={1em,between origins},
              ampersand replacement=\&]%
     {
      \node[commit] (E 9) {9}; \\
      \node[commit] (E 8) {8}; \\
      \node[commit] (E 7) {7}; \\
      \node[commit] (E 6) {6}; \\
      \node[commit] (E 5) {5}; \\
     };
     \node[anchor=south, align=center, font={\scriptsize}] at (E 9.north)
      (E repository) {Local\\Repository};
     \foreach \x in {6,...,9} {
      \pgfmathparse{int(\x-1)}
      \connect{E \pgfmathresult}{E \x};
     }
     \path[connections, densely dotted] (E 5.south) -- +(0,-0.5em);
    \end{scope}
    \begin{scope}[
      every node/.style={anchor=west, minimum height=1em,
                         font={\tiny\ttfamily}, inner sep=2.5pt},
      every edge/.style={thick},
      root/.style={},
      grow via three points={one child at (0.25,-0.75em) and
                             two children at (0.25,-.75em) and (0.25,-1.25em)},
      edge from parent path={(\tikzparentnode.south) |- (\tikzchildnode.west)}
     ]
     \node[anchor=west, font={\scriptsize\sffamily}, align=center, xshift=1em]
      at (E repository.east) (E work) {Working\\Directory};
     \node [root, anchor=north] at (E work.south) (E root) {specfem3d\_globe}
      child { node {DATA} }
      child { node {doc} }
      child { node {EXAMPLES} }
      child { node {setup} }
      child { node {src} }
      child { node {utils} }
      child { node {\dots} };
    \end{scope}
    \node[anchor=south, shift={(0.5em,1em)}] at (E repository.east)
     {Elliott's Computer};
    \draw[->] (E 9.east) to[out=0,in=180] (E root.west);
   \end{tikzpicture}

   \hfill
   \begin{tikzpicture}[%
     remember picture,
     show background rectangle
    ]
    \begin{scope}[
      commit/.style={draw, circle, fill=solarizedRebase02,
                     inner sep=0.5pt, minimum width=0.5em,
                     font={\tiny\sffamily}},
      connections/.style={draw}
     ]
     \matrix [column sep={1em,between origins}, row sep={1em,between origins},
              ampersand replacement=\&]%
     {
      \node[commit] (P 9) {9}; \\
      \node[commit] (P 8) {8}; \\
      \node[commit] (P 7) {7}; \\
      \node[commit] (P 6) {6}; \\
      \node[commit] (P 5) {5}; \\
     };
     \node[anchor=south, align=center, font={\scriptsize}] at (P 9.north)
      (P repository) {Local\\Repository};
     \foreach \x in {6,...,9} {
      \pgfmathparse{int(\x-1)}
      \connect{P \pgfmathresult}{P \x};
     }
     \path[connections, densely dotted] (P 5.south) -- +(0,-0.5em);
    \end{scope}
    \begin{scope}[
      every node/.style={anchor=west, minimum height=1em,
                         font={\tiny\ttfamily}, inner sep=2.5pt},
      every edge/.style={thick},
      root/.style={},
      grow via three points={one child at (0.25,-0.75em) and
                             two children at (0.25,-0.75em) and (0.25,-1.25em)},
      edge from parent path={(\tikzparentnode.south) |- (\tikzchildnode.west)}
     ]
     \node[anchor=west, font={\scriptsize\sffamily}, align=center, xshift=1em]
      at (P repository.east) (P work) {Working\\Directory};
     \node [root, anchor=north] at (P work.south) (P root) {specfem3d\_globe}
      child { node {DATA} }
      child { node {doc} }
      child { node {EXAMPLES} }
      child { node {setup} }
      child { node {src} }
      child { node {utils} }
      child { node {\dots} };
    \end{scope}
    \node[anchor=south, shift={(0.5em,1em)}] at (P repository.east)
     {Piero's Computer};
    \draw[->] (P 9.east) to[out=0,in=180] (P root.west);
   \end{tikzpicture}
  \end{column}
 \end{columns}

 \begin{tikzpicture}[
   remember picture, overlay,
   path/.style={thick, ->},
   label/.style={decorate, decoration={text along path, text align=center,
                                       text color=solarizedRebase0,
                                       raise=1pt,
                                       text={remote}}}
  ]
  \draw<1>[path] (P repository.west) to[out=180,in=0] (repository);
  \draw<1>[label] (repository) to[out=0,in=180] (P repository.west);
  \draw<1>[path] (E repository.west) to[out=180,in=0] (repository);
  \draw<1>[label] (repository) to[out=0,in=180] (E repository.west);
  \draw<2>[path,<->] (P repository.west) to[out=180,in=180] (E repository.west);
  \draw<2>[label, decoration={raise=-1em}]
   (E repository.west) to[out=180,in=180] (P repository.west);
 \end{tikzpicture}
\end{frame}

\begin{frame}[fragile]
 \frametitle{Git Basics}
 \framesubtitle{Configuration}

 \begin{block}{User Name}
  \begin{semiverbatim}
\command{git config -{}-global user.name 'User Name'}
\command{git config -{}-global user.email 'email@example.com'}
\end{semiverbatim}
 \end{block}
 \pause
 \begin{block}{Conveniences}
  \begin{semiverbatim}
\command{git config -{}-global color.ui auto}
\command{git config -{}-global push.default simple}
\end{semiverbatim}
 \end{block}
\end{frame}
