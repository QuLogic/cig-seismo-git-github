\section{Local Development}

\begin{frame}[plain,noframenumbering]
 \vfill
 \begin{center}
  \LARGE \color{solarizedAccent} Local Development
 \end{center}
 \vfill
\end{frame}

\begin{frame}
 \frametitle{Local Development}
 \framesubtitle{Forking}

 \begin{itemize}
  \item Push access to Geodynamics repos on GitHub restricted
  \item How to share code?
  \pause
  \item Create a \alert{fork} (personal GitHub copy of repo):
   \begin{tikzpicture}
    \node[anchor=south west, inner sep=0] at (0,0) {
     \includegraphics[width=\linewidth, height=\textheight, keepaspectratio,
                      trim={0 700px 0 0}, clip]{github}
    };
    \draw[solarizedAccent, thick, rounded corners]
     (9.1,0.3) rectangle (10.0,0.6);
   \end{tikzpicture}
   \begin{columns}
    \begin{column}{0.4\framewidth}
     \begin{tikzpicture}[
       remember picture,
       show background rectangle,
       commit/.style={draw, rectangle, rounded corners, fill=solarizedRebase02,
                      inner sep=1pt, minimum width=1.5em},
       connections/.style={draw}
      ]
      \matrix [column sep={1em,between origins},
               row sep={1.5em,between origins},
               ampersand replacement=\&]%
      {
       \node[commit] (9) {9}; \\
       \node[commit] (8) {8}; \\
       \node[commit] (7) {7}; \\
       \node[commit] (6) {6}; \\
       \node[commit] (5) {5}; \\
       \ghost{4}; \\
      };
      \node[anchor=south, align=center] (CIG repository) at (9.north)
       {Geodynamics\\Repository};
      \foreach \x in {6,...,9} {
       \pgfmathparse{int(\x-1)}
       \connect{\pgfmathresult}{\x};
      }
      \path[connections, densely dashed] (4) -- (5);
     \end{tikzpicture}
    \end{column}
    \begin{column}{0.3\textwidth}
     \begin{tikzpicture}[
       remember picture,
       show background rectangle,
       commit/.style={draw, rectangle, rounded corners, fill=solarizedRebase02,
                      inner sep=1pt, minimum width=1.5em},
       connections/.style={draw}
      ]
      \matrix [column sep={1em,between origins},
               row sep={1.5em,between origins},
               ampersand replacement=\&]%
      {
       \node[commit] (9) {9}; \\
       \node[commit] (8) {8}; \\
       \node[commit] (7) {7}; \\
       \node[commit] (6) {6}; \\
       \node[commit] (5) {5}; \\
       \ghost{4}; \\
      };
      \node[anchor=south, align=center] (My repository) at (9.north)
       {Personal\\Repository};
      \foreach \x in {6,...,9} {
       \pgfmathparse{int(\x-1)}
       \connect{\pgfmathresult}{\x};
      }
      \path[connections, densely dashed] (4) -- (5);
     \end{tikzpicture}
    \end{column}
   \end{columns}

   \begin{tikzpicture}[
     remember picture, overlay,
     path/.style={thick, ->},
     label/.style={decorate, decoration={text along path, text align=center,
                                         text color=solarizedRebase0,
                                         raise=1pt,
                                         text={Fork}}}
    ]
    \draw[path] (CIG repository) to[out=0,in=180] (My repository);
    \draw[label] (CIG repository) to[out=0,in=180] (My repository);
   \end{tikzpicture}
 \end{itemize}
\end{frame}

\begin{frame}[fragile]
 \frametitle{Local Development}
 \framesubtitle{Forking}

 Set remotes for both repos (conventions):
 \begin{itemize}
  \item \texttt{origin} to your fork
  \item \texttt{upstream} to Geodynamics repo
 \end{itemize}

 \begin{exampleblock}{Setting Remotes}
  \begin{semiverbatim}
\comment{Clone \textit{your} fork}
\command{git clone -{}-recursive \\
   https://github.com/\alert{username}/specfem3d_globe.git}

\comment{Add remote for Geodynamics repo}
\command{git remote add upstream \\
   https://github.com/\alert{geodynamics}/specfem3d_globe.git}
\command{git fetch upstream}
\end{semiverbatim}
 \end{exampleblock}
\end{frame}

\begin{frame}[fragile,t]
 \frametitle{Using CIG Code from Git}
 \framesubtitle{Feature Branches}

 \begin{itemize}
  \item Short-lived branches that contain work on a single feature
  \item Recommended (but not required) development workflow
  \item Completely local (unless explicitly shared)
  \item Allows you to switch between work topics quickly
 \end{itemize}

 \begin{minipage}[t][0.45\textheight][t]{0.97\textwidth}
 \vspace{-1em}
 \begin{exampleblock}<only@1>{Creating a Feature Branch}
  \vspace{-1em}
  \begin{semiverbatim}\small
\comment{Create and switch to branch feature-one}
\command{git checkout -b feature-one devel}
Switched to branch `feature-one'

\comment{Switch back to branch devel}
\command{git checkout devel}
Switched to branch `devel'
\end{semiverbatim}
 \end{exampleblock}

 \begin{exampleblock}<only@2>{Deleting a Feature Branch}
  \vspace{-1em}
  \begin{semiverbatim}\small
\comment{Switch back to branch devel}
\command{git checkout devel}
Switched to branch `devel'

\comment{Delete branch feature-one}
\command{git branch -d feature-one}
Deleted branch feature-one (was 6106a0c).
\end{semiverbatim}
 \end{exampleblock}

 \begin{exampleblock}<only@3>{Listing Branches}
  \vspace{-1em}
  \begin{semiverbatim}\small
\command{git branch}
* \hiGreen{devel}
  master
\end{semiverbatim}
 \end{exampleblock}

 \begin{exampleblock}<only@4>{Listing All Branches (including remotes)}
  \vspace{-1em}
  \begin{semiverbatim}\small
\command{git branch -a}
* \hiGreen{devel}
  master
  \hiRed{remotes/origin/devel}
  \hiRed{remotes/origin/master}
  \hiRed{remotes/upstream/devel}
  \hiRed{remotes/upstream/master}
\end{semiverbatim}
 \end{exampleblock}
 \end{minipage}
\end{frame}

\begin{frame}[fragile]
 \frametitle{Local Development}
 \framesubtitle{Reviewing Changes}

 \begin{exampleblock}{Checking Status}
  \vspace{-1em}
  \begin{semiverbatim}
\command{git status}
On branch devel

Changes not staged for commit:
  (use "git add <file>..." to update what will be
   committed)
  (use "git checkout -{}- <file>..." to discard changes
   in working directory)
    \hiRed{modified:   DATA} (new commits)
    \hiRed{modified:   src/specfem3D/specfem3D.F90}

no changes added to commit (use "git add" and/or "git
commit -a")
\end{semiverbatim}
 \end{exampleblock}
\end{frame}

\begin{frame}[fragile]
 \frametitle{Local Development}
 \framesubtitle{Reviewing Changes}

 \begin{exampleblock}{Viewing Diffs}
  \vspace{-1em}
  \begin{semiverbatim}
\command{git diff \only<2->{master} \only<3>{-{}- src/}}\uncover<3>{
\textbf{\color{solarizedBase1}diff -{}-git a/src/specfem3D/specfem3D.F90 b/src/..
index 3ba52f2..2b619bf 100644
-{}-{}- a/src/specfem3D/specfem3D.F90
+++ b/src/specfem3D/specfem3D.F90}
\hiBlue{ @@ -466,6 +466,6 @@}
   ! sets up and precomputes simulation arrays
   call prepare_timerun()

\hiRed{-  ! steps through time iterations}
\hiGreen{+  ! Test change}
   if( UNDO_ATTENUATION ) then
     call iterate_time_undoatt()}
\end{semiverbatim}
 \end{exampleblock}
\end{frame}

\begin{frame}
 \frametitle{Local Development}
 \framesubtitle{Working with the Index}

 \begin{itemize}
  \item \alert{Index} or \alert{staging area} stores changes to be committed

  \begin{tikzpicture}[%
    remember picture,
    show background rectangle,
    background rectangle/.style={draw, visible on=<2->},
    visible on=<2->
   ]
   \begin{scope}[
     commit/.style={draw, rectangle, rounded corners, fill=solarizedRebase02,
                    inner sep=1pt, minimum height=0.75em, minimum width=1em,
                    font={\tiny\sffamily}},
     connections/.style={draw}
    ]
    \matrix [column sep={1em,between origins}, row sep={1em,between origins},
             ampersand replacement=\&]%
    {
     \node[commit,visible on=<5->] (10) {\alert<5>{10}}; \\
     \node[commit] (9) {9}; \\
     \node[commit] (8) {8}; \\
     \node[commit] (7) {7}; \\
     \node[commit] (6) {6}; \\
     \node[commit] (5) {5}; \\
     \ghost{4}; \\
    };
    \node[anchor=south, align=center, font={\scriptsize}] at (10.north)
     (repository) {Local\\Repository};
    \foreach \x in {6,...,9} {
     \pgfmathparse{int(\x-1)}
     \connect{\pgfmathresult}{\x};
    }
    \path[connections, densely dotted] (4) -- (5);
    \path[connections, visible on=<5->] (9) -- (10);
   \end{scope}
   \begin{scope}[
     every node/.style={anchor=west, minimum height=1em,
                        font={\tiny\ttfamily}, inner sep=2.5pt},
     every edge/.style={thick},
     root/.style={},
     grow via three points={one child at (0.25,-0.75em) and
                            two children at (0.25,-0.75em) and (0.25,-1.25em)},
     edge from parent path={(\tikzparentnode.south) |- (\tikzchildnode.west)}
    ]
    \node[anchor=west, font={\scriptsize\sffamily}, align=center, xshift=3em]
     at (repository.east) (index) {Index};
    \node [root, anchor=north, visible on=<4-5>] at (index.south) (index root)
     {src};
   \end{scope}
   \begin{scope}[
     every node/.style={anchor=west, minimum height=1em,
                        font={\tiny\ttfamily}, inner sep=2.5pt},
     every edge/.style={thick},
     root/.style={},
     grow via three points={one child at (0.25,-0.75em) and
                            two children at (0.25,-0.75em) and (0.25,-1.25em)},
     edge from parent path={(\tikzparentnode.south) |- (\tikzchildnode.west)}
    ]
    \node[anchor=west, font={\scriptsize\sffamily}, align=center, xshift=3em]
     at (index.east) (work) {Working\\Directory};
    \node [root, anchor=north] at (work.south) (work root) {specfem3d\_globe}
     child { node {DATA} }
     child { node {doc} }
     child { node {EXAMPLES} }
     child { node {setup} }
     child { node (work src) {\alert<3-4>{src}} }
     child { node {utils} }
     child { node {\dots} };
   \end{scope}
   \node[anchor=south, shift={(0em,1em)}] at (index.north)
    {Information Flow};

   \begin{scope}[
     path/.style={->},
     label/.style={decorate, decoration={text along path, text align=center,
                                         text color=solarizedRebase0,
                                         raise=1pt}}
    ]
    \draw<2>[path] (9.east) to[out=0,in=180] (work root.west);
    \draw<2>[label, decoration={text={|\tiny| git checkout}}]
     (9.east) to[out=0,in=180] (work root.west);
    \draw<4>[path] (work src.west) to[out=180,in=0] (index root.east);
    \draw<4>[label, decoration={text={|\tiny| git add}}]
     (index root.east) to[out=0,in=180] (work src.west);
    \draw<5>[path] (index root.west) to[out=180,in=0] (10.east);
    \draw<5>[label, decoration={text={|\scriptsize| git commit}}]
     (10.east) to[out=0,in=180] (index root.west);
    \draw<6->[dashed] (work root.west) to[out=180,in=0] (10.east);
   \end{scope}
  \end{tikzpicture}
  \item<7-> Option \texttt{-{}-cached} makes commands work in reference to index
  \begin{itemize}
   \item \texttt{git diff} compares index with working directory
   \item \texttt{git diff -{}-cached} compares HEAD with index
  \end{itemize}
 \end{itemize}
\end{frame}

\begin{frame}[fragile]
 \frametitle{Local Development}
 \framesubtitle{Working with the Index}

 \begin{exampleblock}{Adding New Files}
  \begin{semiverbatim}
\command{git add <new file>}
\end{semiverbatim}
 \end{exampleblock}

 \begin{exampleblock}<2->{Removing \textit{Existing} Files}
  \begin{semiverbatim}
\command{git rm <file>}
\end{semiverbatim}
 \end{exampleblock}

 \begin{exampleblock}<3>{Removing \textit{Newly Added} Files (i.e., from index)}
  \begin{semiverbatim}
\command{git rm -{}-cached <file>}
\end{semiverbatim}
 \end{exampleblock}
\end{frame}

\begin{frame}[fragile]
 \frametitle{Local Development}
 \framesubtitle{Working with the Index}

 \begin{exampleblock}{Adding Changes in Files}
  \begin{semiverbatim}
\comment{Add changes from file, patch-by-patch}
\command{git add -{}-patch} \comment{or -p}
\pause
\comment{Add entire change of file (not recommended)}
\command{git add <modified file>}
\pause
\comment{Add all changes of{\em tracked} files (not recommended)}
\command{git add -{}-all}
\end{semiverbatim}
 \end{exampleblock}
\end{frame}

\begin{frame}[fragile]
 \frametitle{Local Development}
 \framesubtitle{Working with the Index}

 \begin{exampleblock}{Undo Changes to Files}
  \begin{semiverbatim}
\comment{Remove changes to files}
\comment{(will match index)}
\command{git checkout -{}- <files>}
\pause
\comment{Remove changes from index}
\comment{(working directory unaltered)}
\command{git reset \uncover<3->{-{}- <files>}}
\pause\pause
\comment{Remove changes from files{\em and} index}
\command{git reset -{}-hard}
\end{semiverbatim}
 \end{exampleblock}
\end{frame}

\begin{frame}[fragile]
 \frametitle{Local Development}
 \framesubtitle{Committing}

 \begin{exampleblock}{Committing Changes in Index}
  \begin{semiverbatim}
\command{git commit}
\end{semiverbatim}
 \end{exampleblock}

 \begin{block}{Commit Messages}
\begin{verbatim}
Capitalized, short (50 chars or less) summary

More detailed explanatory text, if necessary.  Wrap
it to about 72 characters or so.  The blank line
separating the summary from the body is critical
(unless you omit the body entirely); tools like
rebase can get confused if you run the two together.
\end{verbatim}
 \end{block}
\end{frame}

\begin{frame}[fragile,t]
 \frametitle{Using CIG Code from Git}

 \begin{itemize}
  \item Merging may cause \alert{conflicts} --- when upstream changes lines you
        have also changed
  \item Produces conflict markers \alert{<{}<{}<===>{}>{}>} in conflicting files
  \item Status shows conflicting files
  \item Must be added to index to resolve conflict
 \end{itemize}

 \begin{minipage}[t][0.5\textheight][t]{0.97\textwidth}
  \begin{exampleblock}<only@1>{Conflict Status\textcolor{solarizedRebase02}{g}}
   \begin{semiverbatim}
\command{git status}
...
 Unmerged paths:
    (use "git add ..." to mark resolution)

  \hiRed{both modified:   src/specfem3D/specfem3D.F90}
\end{semiverbatim}
  \end{exampleblock}

  \begin{exampleblock}<only@2->{Resolving Conflicts}
   \begin{semiverbatim}
\command{git mergetool \only<3->{-{}-tool=meld}}

\only<4>{\comment{Permanent tool setting}}
\only<4>{\command{git config -{}-global merge.tool meld}}
\end{semiverbatim}
  \end{exampleblock}
 \end{minipage}
\end{frame}

\begin{frame}
 \frametitle{Submodules (or, ``Why git clone -{}-recursive?'')}

 \begin{itemize}
  \item Completely \alert{independent} repositories referenced in another
  \item Changes, commits, etc.{} are local to submodule
  \item Parent just sees a \textit{single} directory --- always treat it as such
  \item \texttt{git fetch} in parent will automatically fetch submodules
  \item \texttt{git fetch} in submodule will \textit{not} fetch parent
  \item \texttt{git submodule update} to keep submodule in sync with parent's
        expectations
 \end{itemize}
\end{frame}

\begin{frame}
 \frametitle{Submodules}
 \framesubtitle{Committing}

 \begin{itemize}
  \item Specfem1d is only repository that doesn't use submodules
  \item In global code, \texttt{DATA}, \texttt{EXAMPLES}, and \texttt{m4} are
        submodules
  \item Not hosted on GitHub, but on CIG servers
   \begin{itemize}
    \item Must obtain push access permission from them
    \item Must change remote URL to use SSH key (see
\href{https://github.com/geodynamics/specfem3d/wiki/Git-tips-for-developers%
\#modifying-the-examples}{wiki})
   \end{itemize}
  \item Changes must be committed ``twice''
   \begin{enumerate}
    \item Commit changes to submodule (i.e., \texttt{DATA}, \texttt{EXAMPLES},
          or \texttt{m4})
    \item Commit new ID in parent
   \end{enumerate}
 \end{itemize}
\end{frame}
